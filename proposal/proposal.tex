\documentclass[11pt]{article}

\usepackage{a4wide,parskip,times}
\usepackage[backend=bibtex]{biblatex}
\addbibresource{/Users/brucecollie/Documents/Uni/Academic/zotero.bib}

\begin{document}

\centerline{\Large Statically Checked Assertions for TESLA}
\vspace{2em}
\centerline{\Large \emph{A Part III project proposal}}
\vspace{2em}
\centerline{\large B. S. Collie (\emph{bsc28}), Trinity Hall}
\vspace{1em}
\centerline{\large Project Supervisor: Dr. Robert Watson}
\vspace{1em}

\begin{abstract}
  Traditional software instrumentation methods are ``instantaneous''---they only deal with
  program state at a particular point of the execution. TESLA \cite{anderson_tesla:_2014}
  provides a library for dynamically checking temporal properties of C programs, allowing
  for more thorough assertions about program behaviour to be made. These assertions are
  made inline within a program through the use of a language extension implemented using
  Clang and LLVM, and are then compiled into automata that provide runtime checking and
  instrumentation. I propose an extension of the TESLA library to allow for static
  checking of such properties, with the aim of decreasing the runtime overhead of this
  kind of instrumentation.
\end{abstract}

\section{Introduction, approach and outcomes}

In their 2014 paper, \textcite{anderson_tesla:_2014} present a method for dynamically
checking the behaviour of C programs. TESLA (``\textit{Temporally Enhanced System Logic
Assertions}'') allows for assertions about program behaviour in the past or the future to
be made---for example, asserting that a particular function had been called successfully
prior to reaching an assertion site. These assertions are written inline using an
extension to the C language implemented with Clang and LLVM, and are evaluated purely at
runtime.

In this project I propose to investigate static checking and analysis of TESLA assertions.
In particular, I aim to characterise the subset of existing TESLA assertions that can be
checked statically (and the accuracy / precision with which this characterisation can be
made). To do this I will look primarily at how the constructed TESLA automata can be
statically checked against representations of the code. The most feasible implementation
strategy here is to write LLVM passes that analyse control and data flow in the IR, then
remove or modify the generated TESLA automata code based on the static properties of the
code. These passes will be written with the aim of analysing a particular subset of checks
with respect to some property---a simple example is that removing checks that a function
has been called in the past is possible if every path through the control flow graph
includes a call to that function before the assertion site is reached. As well as the LLVM
IR, similar analysis may well be possible on other representations (such as the Clang
AST).

In the original TESLA paper, the authors provide several practical uses of TESLA along
with performance analyses. As part of my research, I will aim to reproduce these
experimental results so that the performance impact of my work can be properly evaluated.
Doing this will involve updating the TESLA codebase to use current versions of the
libraries used to implement it (primarily LLVM and Clang). I have verified with the
original authors that this update work is feasible and will not consume an undue amount of
project time. A deliverable at this stage is a version of TESLA that can be built with
up-to-date dependencies.

Performing runtime checks incurs a performance penalty---in a FreeBSD kernel instrumented
using TESLA, slowdown from $1.1-7\times$ was experienced, with the worst performance on
microbenchmarks. Once the dependencies have been updated, my next aim will be to
instrument and verify properties about a piece of software that is currently infeasible to
validate because of this runtime impact. Two potential areas of investigation here
(identified in the original paper) are lock assertions and the TCP implementation in the
FreeBSD kernel. Building such a verification may require extension of the TESLA syntax to
better express the required behaviour (for example, more explicit description of a state
machine). The deliverable at this stage will be this instrumented software together with a
demonstration that static analysis entails a performance improvement.

Evaluating the performance impact of my modifications to TESLA will involve examining both
the compile-time and runtime---to do this, suitable evaluation workloads will need to be
chosen. For example, to evaluate the impact of TESLA on lock-checking overhead, a workload
with high concurrent contention would be necessary. As the software being instrumented
will have long compilation times (partly due to the size of the codebase, and partly due
to TESLA overheads), evaluation will most likely be performed on dedicated server
hardware, as was the case in the original paper.

\section{Workplan}

\begin{description}
  \item[Block 1 (28 Nov - 4 Dec)] Begin bringing TESLA up to date with the current
    versions of LLVM and Clang. Become familiar with TESLA assertion language and runtime
    by instrumenting small programs. Investigate reproduction of initial experimental
    performance results.

  \item[Block 2 (5 - 18 Dec)] Finish gathering reproduced performance data from the
    previously instrumented libraries, identifying any problems with the updated version
    of TESLA if they arise. Begin investigating possible candidates for a new library to
    instrument.

  \item[Block 3 (19 Dec - 1 Jan)] Identify where TESLA can be used to formalise internal
    assumptions made by the library identified in the previous block (e.g.\ implicit state
    machine code that can be made more explicit). Less work scheduled for this block due
    to the festive period.

  \item[Block 4 (2 - 15 Jan)] Instrument library based on previous block's work, and
    gather performance data as before. Begin writing initial static analysis passes to
    optimise existing TESLA automata code where properties can be verified appropriately.

  \item[Block 5 (16 - 29 Jan)] Continue investigating static analysis techniques, along
    with evaluation of their effect on execution and compilation performance of the
    libraries used so far for evaluation. 

  \item[Block 6 (30 Jan - 12 Feb)] Further work on more sophisticated static analysis
    techniques, along with the associated performance analyses. A potential challenge at
    this stage is investigating and dealing with false positives / negatives given by the
    static analysis.

  \item[Block 7 (13 - 26 Feb)] Begin investigating how new TESLA syntax could be used to
    improve the instrumentation added so far. Look into how new assertion syntax can be
    added to TESLA, with a focus on fine-grained assertions that can be statically checked
    effectively (such as state machine code or lock checking).

  \item[Block 8 (27 Feb - 12 Mar)] Continue implementation of new assertion syntax.
    Identify a piece of software that would have been infeasible to instrument previously
    (e.g.\ a TCP implementation), and demonstrate why this is the case by comparing
    instrumented with uninstrumented performance. As before, characterising false
    positives / negatives in the static analysis is important at this stage.

  \item[Block 9 (13 - 26 Mar)] Develop examples showing performance gains through static
    analysis and new assertion syntax. Continue to work on instrumentation and performance
    analysis using the new assertions.

  \item[Block 10 (27 Mar - 9 Apr)] Collate results from newly instrumented software (for
    example, any latent bugs that have been found or formalisms added to the code). Time
    for remaining development if necessary.

  \item[Block 11 (10 - 23 Apr)] Finalise any remaining development and analysis work.
    Start to structure dissertation and gather results together.
  \item[Block 12 (24 Apr - 7 May)] Writing up.
  \item[Block 13 (8 - 21 May)] Writing up.
  \item[Block 14 (22 May - 4 June)] \textit{Contingency}
\end{description}

\newpage
\appendix

\printbibliography

\end{document}
