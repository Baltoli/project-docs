In this report, I have investigated the use of static analysis techniques for
optimising TESLA assertions. My implementation of a model checker for TESLA
(TMC) is able to correctly check a useful subset of its assertions.
Additionally, I have shown significant performance improvements that allow TESLA
to be used in contexts it would not previously have been feasible to.

As well as my work on static analysis, I have contributed to the understanding
and usage of TESLA in general---TMC can be used to produce counterexamples to a
TESLA assertion, and I provide a collection of programs that demonstrate how
TESLA can be used to model temporal properties of a data structure. Future work
on TESLA (whether in the context of static analysis or otherwise) will benefit
from these improvements, as well as from the improvements to the TESLA
implementation and documentation I have contributed.

TMC itself constitutes an explicit formalisation of the underlying principles of
TESLA, as well as demonstrating a novel SMT-based algorithm for proving
properties of control flow that depend partially on data flow. I have identified
ways in which future work could extend these techniques to prove stronger
properties.

An initial goal of the project was to provide a TESLA model of a network
protocol implementation. While this goal was not fully successful, the insights
gained from the investigation itself are a useful contribution in their own
right---I have described features of source code that are hostile to TESLA
instrumentation and static analysis, and introduced a characterisation of
scenarios in which TESLA can be usefully applied.

Motivated by this characterisation of the effective use of TESLA, I have
demonstrated a framework by which library developers can more reliably enforce
temporal properties on usages of their library code. This is a more general
application of TESLA than previous work has demonstrated. The performance
improvements made possible through the use of static analysis are an important
part of this contribution---I have shown that even a small number of assertions
can cause a large runtime performance overhead on instrumented programs, and
that TMC is able to completely eliminate this overhead in real programs.
