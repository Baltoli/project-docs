In this chapter I give a summary of previous work related to TESLA on which this
project builds, a short overview of the practical issues associated with using
TESLA and an introduction to the TESLA assertion language used throughout this
report. I also give an overview of the theoretical background to important
techniques used in the implementation of this project.

\section{Summary of Existing Work}

\begin{displaycquote}[p. 1]{anderson_tesla:_2014}
TESLA is a description, analysis, and validation tool that allows systems
  programmers to describe expected temporal behaviour in low-level languages
  such as C.
\end{displaycquote}

\textcite{anderson_tesla:_2014} introduce TESLA as a tool for validating
safety\footnote{A \emph{safety} property asserts that ``bad things'' do not
happen during the execution of a program, while a \emph{liveness} property
asserts that ``good things'' do eventually happen
\cite{lamport_proving_1977,alpern_defining_1984}.} properties of systems code.
These safety properties are written inline to the program they describe, and are
checked at runtime by means of an extended compilation process that adds
instrumentation to the intermediate representation of a program.

In this paper, the authors contribute a toolchain for adding TESLA
instrumentation to a program (and the associated runtime implementation), a
number of case studies to which TESLA was fruitfully applied and a discussion of
the runtime and compile-time performance impact of TESLA.

\section{Programming with TESLA}

\subsection{Build Process}

Using TESLA to instrument a program requires that the build process for the
program is modified in order to produce the intermediate products required by
the TESLA toolchain. \autoref{fig:c-compilation} shows the traditional
compilation model for C programs, while \autoref{fig:c-tesla-compilation} shows
the additional steps required by TESLA.

\begin{figure}[ht]
  \centering
  \begin{tikzpicture}
    \node [msourcefile] (source1) {\texttt{.c}};

    \node [msourcefile, right=3em of source1] (obj1) {\texttt{.o}};

    \draw[-latex] (source1.east) -- (obj1.west);

    \node [right=3em of obj1] (exe) {executable};

    \draw[-latex] (obj1.east) -- (exe.west);
  \end{tikzpicture}
  \caption{The traditional C compilation model}
  \label{fig:c-compilation}
\end{figure}

\begin{figure}[ht]
  \centering
  \begin{tikzpicture}
    \node [msourcefile] (source1) {\texttt{.c}};

    \node [right=5em of source1] (mid) {};
    \node [msourcefile, above=1em of mid] (bc1) {\texttt{.bc}};
    \node [msourcefile, below=1em of mid] (tesla1) {\texttt{.tesla}};

    \draw[-latex] (source1.east) to [out=0,in=180] (bc1.west);
    \draw[-latex] (source1.east) to [out=0,in=180] (tesla1.west);

    \node [sourcefile,right=3em of tesla1] (tesla2) {\texttt{.manifest}};
    \node [sourcefile] at (bc1 -| tesla2) (bc2) {\texttt{.bc}};

    \draw[-latex] (bc1.east) to [out=0,in=180] (bc2.west);
    \draw[-latex] (tesla1.east) to [out=0,in=180] (tesla2.west);

    \node [sourcefile,right=12em of mid] (instr) {\texttt{.instr.bc}};

    \draw[-latex] (bc2.east) to [out=0,in=180] (instr.west);
    \draw[-latex] (tesla2.east) to [out=0,in=180] (instr.west);

    \node [right=3em of instr] (exe) {executable};

    \draw[-latex] (instr) -- (exe);
  \end{tikzpicture}
  \caption{The C compilation model with TESLA integrated}
  \label{fig:c-tesla-compilation}
\end{figure}

A \texttt{.bc} file contains LLVM intermediate code in its stable binary format,
and a \texttt{.tesla} file contains a binary representation of the TESLA
assertions parsed from the source file (formatted using Google's protocol
buffers\footnote{\url{https://developers.google.com/protocol-buffers/}}).

The TESLA toolchain is used together with the Clang / LLVM compiler
infrastructure to generate these intermediate artifacts. A brief summary of the
individual TESLA tools used is:

\begin{description}
  \item[\texttt{analyze}] parses TESLA assertions from a C source file and
    outputs them to a \texttt{.tesla} manifest file.
  \item[\texttt{instrument}] adds instrumentation code to a program in LLVM IR
    format based on the data in a TESLA manifest file.
  \item[\texttt{cat}] combines several TESLA assertion manifests together,
    checking for consistency and eliminating redundant definitions.
\end{description}

Using these tools as described in the context of an automated build system
imposes some configuration overhead on the programmer. However, this can be
alleviated somewhat by developing generic scripts for the build systems in
question (for example, GNU
Make\footnote{\url{https://www.gnu.org/software/make/}} or
CMake\footnote{\url{https://cmake.org/}}).

\subsection{Writing Assertions}

Once a project has been set up to build against TESLA, assertions can be written
using the included set of preprocessor macros. These macros provide a convenient
wrapper interface over the TESLA marker functions\footnote{These marker
functions have no definitions---their only use is to appear as the target of a
function call that will be replaced with instrumentation code in a later
compilation phase.}.

\section{The TESLA Assertion Language} \label{sec:assertions}

TESLA assertions express temporal logic properties, although for implementation
reasons they are in fact less expressive than LTL\footnote{Linear Temporal
Logic}. In particular, the ``until'' LTL construct \cite{pnueli_temporal_1977} is
not expressible in TESLA because of the requirement that TESLA automata have
bounded lifetimes \cite{anderson_tesla:_2014}. However, despite their decreased
expressivity, TESLA assertions are still capable of expressing many useful
properties of program behaviour.

Assertions specify temporal relations between \emph{program events}, within a
specified bounding interval. These events can be function calls or returns (with
argument or return values specified), assertion sites or structure field
assignments. An assertion site event occurs when program execution reaches the
location where the assertion was originally written.

The assertion language allows for temporal relations between these events to be
specified. For example, it is possible to assert that \textquote{Bounded by
function \texttt{g}, function \texttt{f} is called exactly once after this
assertion site}. Written using the TESLA macros, this could be expressed as:

\begin{minted}{c}
TESLA_WITHIN(g, TSEQUENCE(
  TESLA_ASSERTION_SITE,
  call(f)));
\end{minted}
