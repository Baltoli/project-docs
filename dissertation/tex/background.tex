In this chapter I give a summary of previous work related to TESLA on which this
project builds, a short overview of the practical issues associated with using
TESLA and an introduction to the TESLA assertion language. I also provide a
guide to the TESLA-specific terminology used throughout the rest of this report.

\section{Temporal Assertions}

Before describing TESLA in detail, it is worth giving a motivating example of
why it is useful. The simplest possible explanation is that it allows the
programmer to make assertions about events that occur in the past and future
execution of a program, rather than just about the current program state.

\autoref{lst:mutex-simple} shows C functions for acquiring and releasing a
mutual exclusion lock.\footnote{Modulo a correct implementation of atomic
compare-and-swap (\mintinline{c}{CAS}).} For a program to make progress, it
should eventually release the lock after it has been acquired.  However, within
\mintinline{c}{lock_acquire}, there is no way of asserting this property using
standard C constructs---the call that releases the lock could be logically
separated from the call that acquires it (for example, a library function could
acquire the lock then depend on user code releasing it).

TESLA allows for \emph{temporal} properties to be expressed.
\autoref{lst:mutex-better} shows the same lock acquisition function, but with a
TESLA assertion enforcing the safety property. The property is in the same
source location as the function it applies to, and is independent of where calls
to \mintinline{c}{lock_acquire} and \mintinline{c}{lock_release} are made. The
remainder of this chapter describes TESLA assertions and the associated tooling
in more detail.

\begin{figure}
  \begin{minted}{c}
void lock_acquire(lock_t *lock) {
  return CAS(&lock->locked, false, true);
}

void lock_release(lock_t *lock) {
  lock->locked = false;
}
  \end{minted}
  \caption{Lock operations without progress property enforced}
  \label{lst:mutex-simple}
\end{figure}

\begin{figure}
  \begin{minted}{c}
void lock_acquire(lock_t *lock) {
  TESLA_WITHIN(main, eventually(
    lock_release(lock)
  ));
  return CAS(&lock->locked, false, true);
}

void lock_release(lock_t *lock) {
  TESLA_WITHIN(main, previously(
    lock_acquire(lock) == 1;
  ));
  lock->locked = false;
}
  \end{minted}
  \caption{Lock operations with progress property enforced using TESLA}
  \label{lst:mutex-better}
\end{figure}

\section{Summary of Existing Work}

\begin{displaycquote}[p. 1]{anderson_tesla:_2014}
TESLA is a description, analysis, and validation tool that allows systems
  programmers to describe expected temporal behaviour in low-level languages
  such as C.
\end{displaycquote}

\textcite{anderson_tesla:_2014} introduce TESLA as a tool for validating
safety\footnote{A \emph{safety} property asserts that ``bad things'' do not
happen during the execution of a program, while a \emph{liveness} property
asserts that ``good things'' do eventually happen
\cite{lamport_proving_1977,alpern_defining_1984}.} properties of systems code.
These safety properties are written inline with the program they describe, and
are checked at run time by dynamic instrumentation code added during an extra
compilation phase.

The authors detail their experiences using TESLA to perform complex debugging on
large, well-known systems software. These efforts were successful---the
discovery of a known a security vulnerability in OpenSSL was reproduced using
TESLA, and an elusive bug in the GNUStep graphics library was diagnosed and
fixed. Additionally, they detail the overhead associated with using TESLA (at
compile and run time) and identify static analysis as a possible future
direction of research.

At the time I began work on this project, the primary components of TESLA were a
parser for assertions written in C programs, a compiler-based instrumentation
tool, and a runtime support library (\texttt{libtesla}). The primary
contribution from my work is a model checker for TESLA assertions.
\autoref{fig:components} shows how these components interact with each other.
Existing TESLA components are highlighted in blue, and my contribution in green.
As well as the model checker, I have contributed bug fixes and other
improvements to the existing TESLA components. 

\begin{figure}
  \tikzstyle{box}=[rounded corners,minimum height=2em,minimum width=7em,draw]
  \tikzstyle{them}=[box,fill=blue!20]
  \tikzstyle{me}=[box,fill=green!20]
  \tikzstyle{other}=[box,fill=gray!20]

  \makebox[\textwidth][c]{
  \begin{tikzpicture}
    [
      >=latex,auto,node distance=2.3cm,
      every label/.append style={font=\footnotesize},
      every edge/.append style={nodes={font=\footnotesize}}
    ]
    \node[them] (ana) {Analyser};
    \node[left=of ana,coordinate] (i) {};
    \path[->] (i) edge node {source code} (ana);

    \node[other,below=of ana] (com) {Compiler};
    \node[left=of com,coordinate] (j) {};
    \path[->] (j) edge node {source code} (com);

    \node[them,right=of ana] (ins) {Instrumenter};
    \path[->] (ana) edge node {assertions} (ins);
    \draw[->] (com.east) to[out=0,in=180] node {LLVM IR} (ins.west);

    \node[me,below=of ins] (mc) {Model Checker};
    \path[<->] (ins) edge node[align=center] {property\\ checks} (mc);

    \node[other,right=of ins] (com2) {Compiler};
    \path[->] (ins) edge node[align=center] {instrumented\\ LLVM IR} (com2);

    \node[them,below=of com2] (lib) {\texttt{libtesla}};
    \path[->] (lib) edge node {link} (com2);

    \node[right=of com2,coordinate] (k) {};
    \path[->] (com2) edge node {program} (k);
  \end{tikzpicture}
  }
  \caption{TESLA system components}
  \label{fig:components}
\end{figure}

\section{Programming with TESLA}

In this section I give a brief overview of how TESLA is used in practice
to instrument programs.

\subsection{Terminology} \label{sec:terminology}

Today, programmers may add \emph{assertions} to their code to ensure its
correctness---these are logical statements predicated on data in the current
scope. TESLA assertions are similar in concept, but express temporal relations
between \emph{program events}. Temporal assertions require a \emph{bounding
interval}---a pair of start and end events that limit the scope of the
assertion. In \autoref{lst:mutex-better}, the bounding interval is from each
call to \mintinline{c}{main} to the corresponding return.

Each TESLA assertion defines an \emph{automaton}, and a collection of
these automata is referred to as a \emph{manifest} when serialised to
disk. An \emph{assertion site} is the source location where an assertion
was originally written, and a \emph{function event} is a call to or
return from a function.

\subsection{Build Process} \label{sec:build-tesla}

Using TESLA to instrument a program requires that its build process is modified
to produce TESLA-specific intermediate products. \autoref{fig:c-compilation}
shows the traditional compilation model for C programs;
\autoref{fig:c-tesla-compilation} shows the additional steps required by TESLA.

\afterpage{%
\clearpage\clearpage
\begin{figure}[t]
  \centering
  \begin{tikzpicture}
    [
      >=latex,auto,node distance=2.3cm,
      every label/.append style={font=\footnotesize},
      every edge/.append style={nodes={font=\footnotesize}}
    ]
    \node [msourcefile] (source1) {\texttt{.c}};

    \node [msourcefile, right=3em of source1] (obj1) {\texttt{.o}};

    \path[->] (source1.east) edge node {\texttt{cc}} (obj1.west);

    \node [sourcefile,right=3em of obj1] (exe) {executable};

    \path[->] (obj1.east) edge node {\texttt{ld}} (exe.west);
  \end{tikzpicture}
  \caption{The traditional C compilation model}
  \label{fig:c-compilation}
\end{figure}

\begin{figure}[h]
  \makebox[\textwidth][c]{
  \begin{tikzpicture}
    [
      >=latex,auto,node distance=2.2cm,
      every label/.append style={font=\footnotesize},
      every edge/.append style={nodes={font=\footnotesize}}
    ]
    \node [msourcefile] (source1) {\texttt{.c}\\ with TESLA\\ assertions};

    \node [right=of source1] (mid) {};
    \node [msourcefile, above=1em of mid] (bc1) {\texttt{.bc}};
    \node [msourcefile, below=1em of mid] (tesla1) {\texttt{.tesla}};

    \path[out=0,in=180,->] (source1.east) edge node {\texttt{cc}} (bc1.west);
    \path[out=0,in=180,->] (source1.east) edge node[below left=1em and -0.5em,align=center] {\texttt{tesla}\\ \texttt{analyse}} (tesla1.west);

    \node [sourcefile,right=of tesla1] (tesla2) {\texttt{.manifest}};
    \node [sourcefile] at (bc1 -| tesla2) (bc2) {\texttt{.bc}};

    \path[->] (bc1.east) edge node {\texttt{llvm-link}} (bc2.west);
    \path[->] (tesla1.east) edge node[align=center] {\texttt{tesla}\\ \texttt{cat}} (tesla2.west);

    \node [sourcefile,right=of ($(bc2)!0.5!(tesla2)$)] (instr) {\texttt{.instr.bc}};

    \path[out=0,in=180,->] (bc2.east) edge node[above right=1em and -0.5em,align=center] {\texttt{tesla}\\ \texttt{instrument}} (instr.west);
    \path[out=0,in=180,->] (tesla2.east) edge node[below right=1em and
    -0.5em,align=center] {\texttt{tesla}\\ \texttt{instrument}} (instr.west);

    \node [sourcefile,right=1cm of instr] (exe) {executable};

    \draw[->] (instr) edge node {\texttt{cc}} (exe);
  \end{tikzpicture}
  }
  \caption{The C compilation model with TESLA}
  \label{fig:c-tesla-compilation}
\end{figure}
}

A \texttt{.bc} file contains LLVM intermediate code, and a \texttt{.tesla} file
contains a binary or textual representation of the TESLA assertion manifest.

The TESLA toolchain is used together with the Clang / LLVM compiler
infrastructure to generate these intermediate artifacts. A brief summary of the
individual TESLA tools used is:

\begin{description}
  \item[\texttt{analyze}] parses TESLA assertions from a C source file and
    outputs them to a \texttt{.tesla} manifest file.
  \item[\texttt{instrument}] adds instrumentation code to a program in LLVM IR
    format based on the data in a TESLA manifest file.
  \item[\texttt{cat}] combines several TESLA assertion manifests together,
    checking for consistency and eliminating redundant definitions.
\end{description}

These tools can be easily integrated with an automatic build system. An issue
specific to TESLA is that assertions written in one compilation unit can affect
instrumentation code in all other compilation units. This means that changing
one source file can cause the entire project to be reinstrumented, increasing
build times. Incremental build times are worst affected by this issue.

\subsection{Writing Assertions}

TESLA assertions are written using a set of preprocessor macros that expand to
calls to stand-in functions. These functions have no definition, and are only
used as a way to store information in the IR. Calls to them are removed by the
instrumenter.

An example of the TESLA macros being written inline with a program is
given in \autoref{lst:tesla-example}.

\begin{figure}
  \begin{minted}{c}
int main(void)
{
  TESLA_WITHIN(main,
    eventually(
      call(some_function(ANY(ptr))),
      other_function(ANY(int)) == 0
    )
  );

  int x;
  some_function(&x);

  return other_function(x);
}
  \end{minted}
  \caption{Example of TESLA macros being used to write an assertion}
  \label{lst:tesla-example}
\end{figure}

\subsection{The TESLA Assertion Language} \label{sec:assertions}

Assertions specify temporal relations between \emph{program events} as defined
in \autoref{sec:terminology} (function calls, assertion sites etc.).

The basic temporal relation expressible in the assertion language is
sequencing---an assertion that events occur in a particular order. Disjunction
of assertions is also expressible. Assertions must include a reference to an
assertion site event---this reference is implicit in \autoref{lst:tesla-example}
through the \mintinline{c}{eventually} macro, which states that the sequence of
events listed take place after the assertion site. The properties expressible
using these constructions are of the form ``in the future, events $x, y, z$
happen in order'' or ``in the past, at least one of events $a, b, c$ happened''.

Automata can be written \emph{explicitly} to allow for composition (similar to
defining a function in an ordinary programming language).
\autoref{lst:tesla-explicit} shows an example of this style.

\begin{figure}
  \begin{minted}{c}
automaton(name_of_auto, struct arg_type *s) {
  some_function(s) == 0;
  call(function1) || call(function2);
  s->field = 4;
  tesla_done;
}
  \end{minted}
  \caption{Example of an explicit TESLA automaton}
  \label{lst:tesla-explicit}
\end{figure}

\section{LLVM}

In this section I give a brief overview of LLVM \cite{lattner_llvm:_2002}, the
compiler framework used to implement the TESLA toolchain. In particular, I
describe its intermediate representation and associated tooling.

The primary attraction of implementing a compiler-based tool using LLVM is its
\emph{intermediate representation}, which can be easily manipulated in-memory
using a C++ API or manipulated as human-readable text in an assembly-like
format. LLVM IR is \textquote{a Static Single Assignment (SSA) based
representation that provides type safety, low-level operations [and]
flexibility}\footnote{\url{http://llvm.org/docs/LangRef.html\#abstract}}.

\begin{figure}
  \begin{minted}{llvm}
declare void @bar(i32)

define i32 @foo(i32 %x) {
entry:
  %mul = mul i32 %x, 3
  %cmp = icmp sgt i32 %mul, 100
  br i1 %cmp, label %if.then, label %if.else

if.then:
  call void @bar(i32 %mul)
  ret i32 0

if.else:
  ret i32 %mul
}
  \end{minted}
  \caption{Example of textual LLVM IR code}
  \label{lst:llvm-assembly}
\end{figure}

\autoref{lst:llvm-assembly} gives an example of textual IR that illustrates some
of the important features of LLVM:
\begin{description}
  \item[Functions] LLVM IR can represent functions that are conceptually similar
    to C functions, though control flow is made explicit by using basic blocks
    and jumps rather than loops and conditional statements. Similarly to C,
    functions can be declared but not defined.

  \item[Values] A key design decision taken by LLVM is that instructions produce
    named SSA values. When manipulating LLVM IR programatically, an instruction
    is synonymous with the value it produces. In textual IR, values are
    identified by \texttt{\%} or \texttt{@} preceding their name (for local and
    global values respectively).

  \item[Types] LLVM IR retains strong type information from its source
    program---all values are typed, and casts between types must be made
    explicitly using special instructions.
\end{description}

The TESLA instrumenter is implemented using the in-memory IR manipulation
libraries provided by LLVM. It replaces known instrumentation sites in the IR
with TESLA-specific instrumentation code that can then be compiled to an
executable.
