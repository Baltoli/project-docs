The tools available for programmers to assert the correctness of their code are
almost always \emph{instantaneous}---assertions can be made about the current
state of the program, but not about previous or future states. As a result,
\emph{temporal} properties of programs are often checked informally or not at
all.

TESLA \cite{anderson_tesla:_2014} provides systems programmers with a means of
mechanically checking temporal properties of their code by modifying the
compilation process to insert instrumentation into programs. This approach
proved successful---a number of bugs in large open-source libraries were
identified and fixed with the help of temporal assertions. However, using TESLA
imposes some runtime performance overhead on a program, and so its usage so far
has been restricted to debugging scenarios where this overhead is acceptable.

In this report I propose the use of \emph{static analysis} for optimisation of
TESLA assertions---if an assertion can be proved correct at compile time, then
its instrumentation code can be omitted from the program. This has a number of
benefits. For example, the program is likely to be smaller and faster than if
the instrumentation were included, and potential counterexamples to assertions
can be given to the programmer as a useful debugging tool in their own right.

I motivate the use of static analysis by providing an implementation of a small
data structure with associated TESLA assertions and benchmarks demonstrating the
runtime overhead of using TESLA.  Then, I demonstrate a set of highly
specialised program analyses that can be used to prove the correctness of these
assertions in particular.  From there, I contribute a translation of TESLA
assertions to finite state automata that formalises components of the original
work. I use this translation to describe an algorithm for checking the
correctness of a program with respect to a checkable subset of TESLA assertions.
Finally, I give an implementation of this algorithm that can be easily
integrated into the existing TESLA toolchain.

To evaluate this model checker, I perform an investigation into how TESLA can be
applied to a production-standard implementation of a network protocol stack
(with emphasis on the potential of statically analysing this instrumentation).
Then, motivated by the difficulties encountered during this process, I
contribute a general framework for library authors to make temporal assertions
about client code using the library. I give an example application written using
this framework and show the large performance benefits available by applying
static analysis.
