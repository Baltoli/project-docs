\begin{center}
  \Large{Statically Checked Assertions for TESLA}
\end{center}

\begin{abstract}

TESLA \cite{anderson_tesla:_2014} is a compiler-based tool that allows
  temporal run-time assertions to be written about C programs,
  expressing richer safety properties than traditional tools can.
  However, adding temporal assertions to a program incurs a significant
  runtime performance overhead. As a result, the usefulness of TESLA is
  limited to debugging scenarios where this overhead is acceptable. 
  
  I formalise the assertion language used by TESLA, then use the
  formalism to develop a \emph{model checker} (TMC) for a subset of its
  assertions, demonstrating its usefulness by applying it to a
  TESLA-instrumented implementation of mutual exclusion locks. I analyse
  the usefulness of TESLA with respect to a large, widely-used open
  source library, finding that common source-code idioms inhibit the
  useful application of TESLA. Based on these results, I improve on
  previous work by describing a general method for applying TESLA to
  library interfaces, then apply the method successfully to an existing
  server application. To evaluate TMC, I analyse the performance
  overhead of TESLA instrumentation in this server, finding a $40\%$
  overhead for as few as five assertions. Applying TMC to prove these
  assertions at compile time eliminates the run-time overhead completely
  while retaining the asserted safety properties.

\end{abstract}
