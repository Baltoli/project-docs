In this chapter I discuss potential applications of statically checked TESLA to
practical software engineering scenarios. I provide an analysis of how coding
style can make writing TESLA assertions for a system more difficult, with
reference to a large open-source library. Then, motivated by this difficulty, I
describe a general method for applying TESLA to library interfaces. Finally, I
demonstrate a practical application of this technique by adding TESLA
instrumentation to the interface of a widely-used TCP implementation.

\section{LWIP} \label{sec:lwip}

An initial goal of the project was to investigate how TESLA might be
applied to verify the behaviour of a larger state machine such as that
of TCP. However, this verification proved to be more difficult than
anticipated for a number of reasons.

In this section I investigate the application of TESLA to LWIP
\cite{dunkels_design_2001}, a widely used, portable implementation of
the IP protocol stack. I describe difficulties encountered in this
process with reference to the LWIP source, as well as an analysis of how
code written from scratch with TESLA instrumentation in mind could
mitigate these issues.

\subsection{Structure}

LWIP is distributed as a configurable library so that it can be built on
virtually any platform with a C compiler---interfaces to network
buffers, timers and other platform-specific code are abstracted so that
their implementation can be specified by users. Configurations for
widely used operating systems (generic Unix, Windows etc.) are
distributed as a secondary library together with example applications.

The core networking code of LWIP is around 57K lines of C.\footnote{Not
including header files or tests.} This code includes implementations of
IPV4, IPV6, TCP, UDP and a number of application-layer protocols. The
secondary library has around 13K lines of C, mostly contained in
implementations of executable server applications (HTTP, Telnet, SMTP
etc.).

\subsection{Investigation}

The goal of my investigation into LWIP was to instrument the core TCP
implementation with useful TESLA assertions, then to demonstrate that
performance improvements were attainable by applying static analysis to
this instrumentation.

Before any investigation could be performed, a version of LWIP built
using the TESLA compilation model described in \autoref{sec:build-tesla}
was required.

Each of the LWIP-based server applications is built using a Makefile
that compiles the core library separately, then links the
application-specific code with the core library. Modifying this build
system to use the TESLA infrastructure was not difficult, as much of the
setup (flags, includes, linking etc.) was in place already---the only
changes needed were to add the extra TESLA-specific rules and to compile
to bitcode instead of object files.

The end result of this modification was that any of the example applications
distributed with LWIP could be instrumented and built against TESLA. Based on
this modified build of LWIP, I investigated how TESLA could be applied to
internal library code. The remainder of this section describes the features
found in the source code that made applying TESLA (statically analysed or
otherwise) more difficult.

\subsubsection{TCP State Implementation}

At the core of the TCP protocol implementation is a structure
representing a single TCP connection (\mintinline{c}{struct tcp_pcb}).
Almost all of the TCP protocol implementation is expressed in terms of
these structures---\autoref{lst:tcp-decls} contains some function
declarations taken from the source code that use the structure.

\begin{figure}
  \begin{minted}{c}
err_t tcp_bind(struct tcp_pcb *pcb, const ip_addr_t *ipaddr, u16_t port);

err_t tcp_close(struct tcp_pcb *pcb);

struct tcp_pcb * tcp_listen_with_backlog(struct tcp_pcb *pcb, u8_t backlog)
  \end{minted}
  \caption{Function declarations from the LWIP TCP implementation.}
  \label{lst:tcp-decls}
\end{figure}

Such heavy reliance on structure fields is obviously less than ideal
from the perspective of static analysis, although checking assignments
to structure fields is within the capabilities of TESLA without static
analysis. However, a further complication is that the implementation is
not consistent in its use of PCB structures---some functions modify a
structure passed to them, while others return an entirely new structure.
The latter style of function is far more difficult to instrument
effectively in TESLA.

\subsubsection{Macro Usage}

In order for LWIP to be universally portable, it makes heavy use of the
C preprocessor for a number of reasons. For example:
\begin{description}
  \item[Platform-specific implementations] The implementation of some
  functions can vary from system to system (e.g.\ endianness conversion
  functions). Macros are used to select the correct implementation of
  these functions without the overhead of a function call. This means
  that any TESLA assertions added to these functions would become
  platform-specific, and potentially duplicated between implementations.

  \item[Conditional Compilation] Almost every feature of LWIP can be
  enabled, disabled or modified at compile-time by setting the correct
  preprocessor definitions (this is what allows LWIP to be used so
  effectively on systems with limited resources). In LWIP, this feature
  is used in places to conditionally change the fields contained in a
  structure---any assertions written about that field must then be aware
  of the required \mintinline{c}{#ifdef} context.

  \item[Inlined Functions] Some simple ``functions'' in LWIP are expressed using
    macros to guarantee that there is no function call overhead, rather than
    relying on the compiler to inline them. These function-like macros cannot be
    asserted about by TESLA\footnote{Because TESLA runs after preprocessing has
    been applied to the source code.}, and are difficult to distinguish in source code.
\end{description}

Taken together, these uses of the macro system make TESLA
instrumentation far more difficult to add to the LWIP source.

\subsubsection{Control Flow}

TESLA assertions are most useful (especially when using static analysis)
for asserting properties related to control flow events. However, the
style in which LWIP code is written means that there is little explicit
control flow within the protocol implementation itself---many functions
perform complicated work on a PCB structure, then call only a single
other function to send a packet.

In addition to the long functions and shallow call graph in the TCP
implementation, applying TESLA becomes even more difficult because of the way
users of the TCP implementation call into it---code that uses the TCP
implementation must register a set of callback functions that are called at
specified points in the protocol's execution. This means that control flow moves
between user and library code through a \emph{dynamic} interface that cannot be
reasoned about with TESLA. \autoref{lst:callbacks} shows an extract from the TCP
echo server in which these callbacks are registered.

\begin{figure}
  \begin{minted}{c}
tcp_recv(newpcb, tcpecho_raw_recv);
tcp_err(newpcb, tcpecho_raw_error);
tcp_poll(newpcb, tcpecho_raw_poll, 0);
tcp_sent(newpcb, tcpecho_raw_sent);
  \end{minted}
  \caption{Callback registration for a user of the LWIP TCP implementation}
  \label{lst:callbacks}
\end{figure}

Registered callbacks are stored as members of a PCB structure. This
behaviour defeats TESLA instrumentation (both static and
dynamic)---there is currently no way to express \textquote{the function
\texttt{pcb.member} is eventually called} in the assertion language.
Unfortunately, these callback functions contain much of the behaviour
that would be well-served by TESLA instrumentation. For example, the
LWIP
documentation\footnote{\url{http://lwip.wikia.com/wiki/Raw/TCP\#Receiving_TCP_data}}
describes the mandated behaviour of a particular callback function:

\begin{displayquote}[LWIP Wiki]
When the application has processed the incoming data, it must call the
\mintinline{c}{tcp_recved()} function to indicate that TCP can increase
the receive window.
\end{displayquote}

Because of the callback interface, instrumenting the invariants of this
function could only be done by the consumer of the library (rather than
the author of the library). This means that the author of the library
can do little beyond documentation to ensure correct usage of the API
functions. In \autoref{sec:safer-libs} I show how this problem can be partially
solved using TESLA.

\subsection{Summary}

The LWIP TCP library presents an interesting target for verification with TESLA.
However, the style in which the library is written means that applying TESLA
assertions to the internal code is both difficult and unlikely to yield any
useful insight into the behaviour of the library. Further informal investigation
into the FreeBSD TCP implementation yielded much the same conclusions, and it is
likely that other similar libraries would suffer the same problems.
Additionally, the use of a callback-based API for users of the library means
that TESLA cannot be directly applied in the situation where it would be most
useful (enforcing temporal assertions on consumer code). An interesting
direction for future work would be to quantify the degree of modification
required for library code to be usefully instrumented with TESLA.

\section{Safer Library Interfaces with TESLA} \label{sec:safer-libs}

In this section I describe the implementation of a mechanism by which a library
can use TESLA assertions to verify correct usage of the library by clients.
First, I relate the problem to the difficulties encountered when attempting to
apply TESLA to LWIP in \autoref{sec:lwip}. Then, I describe the construction of
such an interface using TESLA. Finally, I succesfully apply the technique to a
existing application using the LWIP TCP library.

\subsection{Motivation}

In \autoref{sec:lwip} I investigated how TESLA instrumentation could be applied
to the internal implementation of LWIP, concluding that the most useful place
for instrumentation is actually at the boundary between user and library code.
However, the use of user-registered callbacks means that the library cannot use
TESLA to make assertions on user behaviour.

A solution to this problem should allow users to consume the LWIP libraries as
they would when using the callback API, while also allowing the library to add
TESLA assertions about the behaviour of user code. This would mean the library
is able to enforce temporal safety properties of user programs without any prior
knowledge of the programs.

\subsection{Implementation Strategy}

LWIP distributes a number of example applications that consume the internal TCP
API by using callbacks as described previously---the simplest of these is an
implementation of the echo protocol \cite{RFC0862}. In this section I describe a
modified version of this application that includes TESLA assertions specified by
the library.

In normal usage, a program using the LWIP TCP library calls a library function
with a function pointer argument to register their application-specific
callbacks. This is a very flexible approach that gives the program fine-grained
control over how it interacts with the library. \autoref{fig:callbacks} shows
how the program interacts with a library using this approach.

\begin{figure}
  \centering
  \begin{tikzpicture}
    \node[code] (call1) {\texttt{callback\textsubscript{1}}};
    \node[code,below=of call1] (call2) {\texttt{callback\textsubscript{2}}};
    \node[below=of call2] (calletc) {...};
    \node[code,below=of calletc] (calln) {\texttt{callback\textsubscript{n}}};
    \node[above=of call1] (req) {};
    \node[code,right=of |(req)(calln),span vertical=(req)(calln)] (lib) {\texttt{liblwip}};
    \node[code,left=of |(req)(calln),span vertical=(req)(calln)] (prog) {\texttt{program}};

    \draw[-latex] (prog.east |- req.south) -- (req.south -| lib.west);

    \draw[latex-latex] (call1) -- (call1 -| lib.west);
    \draw[latex-latex] (call2) -- (call2 -| lib.west);
    \draw[latex-latex] (calln) -- (calln -| lib.west);

    \draw[-latex] (call1) -- (call1 -| prog.east);
    \draw[-latex] (call2) -- (call2 -| prog.east);
    \draw[-latex] (calln) -- (calln -| prog.east);
  \end{tikzpicture}
  \caption{Default usage of the LWIP callback API}
  \label{fig:callbacks}
\end{figure}

To add TESLA assertions, user programs must implement a static interface that
can be linked with the library. The primary loss of flexibility with this
approach is that users cannot change their callback functions at compile time.
However, this is not yet a complete solution. Each TESLA assertion must be
placed at a source location on the execution path it asserts over---intuitively,
this would be within the user-supplied interface functions themselves. Because
the definitions of these functions are not available to the library ahead of
time, it must implement wrapper functions. These wrappers call through to the
user-supplied interface functions, as well as containing the library TESLA
assertions.  \autoref{fig:callbacks-tesla} shows how a program interacts with
this modified usage model.

\begin{figure}
  \centering
  \begin{tikzpicture}
    \node[code] (call1) {\texttt{interface\textsubscript{1}}};
    \node[code,below=of call1] (call2) {\texttt{interface\textsubscript{2}}};
    \node[below=of call2] (calletc) {...};
    \node[code,below=of calletc] (calln) {\texttt{interface\textsubscript{n}}};
    \node[above=of call1] (req) {};

    \node[code,right=of call1] (wrap1) {\texttt{wrapper\textsubscript{1}}};
    \node[code,right=of call2] (wrap2) {\texttt{wrapper\textsubscript{2}}};
    \node[below=of wrap2] (wrapetc) {...};
    \node[code,right=of calln] (wrapn) {\texttt{wrapper\textsubscript{n}}};

    \node[inner sep=0.8em,fit=(wrap1)(wrapn)] (tesla) {};

    \node[code,right=of |(req)(wrapn),span vertical=(req)(tesla)] (lib) {\texttt{liblwip}};
    \node[code,left=5em of |(req)(calln),span vertical=(req)(tesla)] (prog) {\texttt{program}};

    \draw[-latex] (prog.east |- req.south) -- (req.south -| lib.west);

    \draw[latex-] (call1) -- (wrap1);
    \draw[latex-] (call2) -- (wrap2);
    \draw[latex-] (calln) -- (wrapn);

    \draw[latex-] (wrap1) -- (call1 -| lib.west);
    \draw[latex-] (wrap2) -- (call2 -| lib.west);
    \draw[latex-] (wrapn) -- (calln -| lib.west);

    \draw[-latex] (call1) -- (call1 -| prog.east);
    \draw[-latex] (call2) -- (call2 -| prog.east);
    \draw[-latex] (calln) -- (calln -| prog.east);

    \draw[-latex] (call1.north) to ($ (call1.north) + (0,0.5) $) to ($ (call1.north -| lib.west) + (0,0.5) $);
    \draw[-latex] (call2.north) to ($ (call2.north) + (0,0.5) $) to ($ (call2.north -| lib.west) + (0,0.5) $);
    \draw[-latex] (calln.north) to ($ (calln.north) + (0,0.5) $) to ($ (calln.north -| lib.west) + (0,0.5) $);
  \end{tikzpicture}
  \caption{Usage of the LWIP callback API with TESLA instrumentation}
  \label{fig:callbacks-tesla}
\end{figure}

Adapting the echo server application to use this model required only that the
wrapper functions and corresponding assertions were written---almost no
modification of the application code was required beyond removing the callback
registration calls.

This adaptation allows two versions of the server to be built---one where static
analysis has been applied to the library assertions, and one where it has not
been.

\subsection{Summary}

In this section I have shown how TESLA can be used by library developers to
apply temporal assertions to users of the library without prior knowledge of the
user code. Additionally, I have demonstrated a method for adapting library
interfaces not explicitly designed with TESLA instrumentation in mind. This
technique was applied successfully to an existing application (with minimal
modification to application code). In \autoref{sec:eval-app} I give a detailed
evaluation of the performance of the instrumented applications.

The methods developed have some shortcomings when compared to typical C library
development:
\begin{description}
  \item[Distribution] A library developed using these methods must be
    distributed as LLVM bitcode together with the associated TESLA
    manifest---this means that users must install the TESLA toolchain and adapt
    their build process to use the library.

  \item[Safety] TESLA has no way of preventing users from performing unsafe
    operations outsidof the library code. For example, if a library asserts that
    a lock is held before a logging function is called, nothing stops the user
    from making calls to \mintinline{c}{fprintf} without holding the lock.

  \item[Performance] If the user code is amenable to static analysis, then the
    performance impact of TESLA is minimal. However, this is dependent on how
    the user code is written.
\end{description}

Despite these issues, the interface adaptation technique is still a useful tool
for library developers to prevent bugs in user code. Additionally, it represents
a significant generalisation of previous applications of TESLA.
