In this chapter I provide an overview of how my work on TESLA contributes new
ideas and developments compared to relevant previous work, with particular
emphasis on the areas of bounded model checking and SMT methods.

\section{Program Verification} \label{sec:bounded-model-checking}

In this section I give a summary of important work in the area of program
analysis and verification that has influenced my development of static analysis
for TESLA.

\subsection{Bounded Model Checking}

\textcite{biere_symbolic_1999} introduce the concept of bounded model checking,
building on the earlier idea of \emph{symbolic} model checking due to
\textcite{mcmillan_symbolic_1992}. In this section I give a summary of previous
work in the area, with particular emphasis on where it has been applied to check
C programs. I also note key differences between these previous works and my
implementation of a TESLA model checker---namely the style of assertion
supported, and the use of non-symbolic checking in my implementation.

At the heart of model checking is the idea of counterexample generation---if a
state is found to satisfy the negation of a formula, then that state is a
counterexample for the original formula. Bounded model checking builds on this
idea by searching for counterexamples with an upper bound on their allowable
size. This means that the counterexamples discovered are minimally sized, and
are discovered faster.

While the implementation strategies and analysis goals of BMC
\cite{biere_symbolic_1999} (the first practical bounded model checker) and the
TESLA model checker are quite different (BMC is a symbolic model checker for
SMV\footnote{A state description language for model checking problems.}; the
TESLA model checker is non-symbolic and checks TESLA assertions embedded in C
programs), the strategy of searching for counterexamples using an
iterative-deepening method is derived in part from the original work on bounded
model checking. Another difference between the TESLA model checker and BMC is
their approaches to soundness---BMC aimed to be provably sound, while TESLA aims
for \emph{soundiness} \cite{livshits_defense_2015} (as long as the upper bound
is large enough for the program in question, the model checker is sound).

\subsubsection{CBMC}

\citeauthor{clarke_behavioral_2003}'s CBMC \cite{clarke_behavioral_2003} applied
bounded model checking to C programs by translating them into instances of the
SAT decision problem. CBMC allowed for C programs to be written as executable
specifications for synchronous Verilog hardware designs without prohibiting the
use of any C programming construct (use of pointers, memory allocations and
recursive calls are all permitted).

Although CBMC is an example of model checking being applied to C programs, the
assertions that can be checked are only local assertions. The scope and target
of TESLA assertions is different to those of CBMC---TESLA assertions are written
to verify the behaviour of a program, while CBMC uses a C program with
assertions to verify that a hardware implementation is behaviourally equivalent
to that program.

\subsubsection{LLBMC}

CBMC operates at the level of C source code using syntactic transformations.
\textcite{merz_llbmc:_2012} identify this as a potential avenue for
improvement---they contribute LLBMC, a bounded model checker that operates on
LLVM \cite{lattner_llvm:_2002} intermediate representation. The benefits of this
approach include broader language support, assistance from compiler
optimisations for code simplification, and an improved memory model.
Empirically, LLBMC represents an improvement over previous work it is compared
to.

The use of a compiler intermediate representation to simplify model
checking is similar in concept to TESLA (modulo the differences between
symbolic and non-symbolic model checking).

\subsubsection{Context-bounded LTL Checking}

More recent work by \textcite{morse_context-bounded_2011,
morse_model_2015-1} uses bounded model checking to verify more complex
LTL assertions made of a C program. Their approach involves the
translation of an LTL formula into a B\"uchi automaton, which is then
itself converted to C code and woven into the program to be checked.

This approach is reminiscent of TESLA instrumentation, but has some key
differences. The automaton code here has its own thread of execution rather than
being inserted inline, and instead of producing a modified executable that
exhibits runtime failures on assertion violations, the combined program is
checked statically using ESMBC \cite{cordeiro_smt-based_2009}.

Although the development of this work was partly contemporaneous with
the initial work on TESLA, and the implementation strategies were
similar in some respects, the LTL assertions checkable by this system
are closer to those asserted by CBMC than to those asserted by TESLA.

\subsection{Other Approaches}

While there has been a great deal of work derived from BMC on verifying
systems software, there are also other approaches that do not share the
same lineage. 

\textcite{bessey_few_2010} describe the lessons learned when commercialising
their static analysis research---many of these lessons are applicable to static
analysis tools in general, particularly with regard to the \emph{perceived
usability} of a tool and how it is used in a non-research environment.

\subsubsection{MOPS}

\textcite{anderson_tesla:_2014} identify MOPS \cite{chen_mops:_2002} as being
partially similar to TESLA in concept. The primary goal of MOPS is to discover
potential vulnerabilities of C programs operating in a Unix environment (where
security properties may have whole papers dedicated to explaining their
subtleties, as is the case with \mintinline{c}{setuid} \cite{chen_setuid_2002}).
The authors were able to use MOPS to discover a number of security
vulnerabilities in well-known open-source software.

Properties in MOPS are expressed as finite state automata, with accepting states
representing an execution on which an unsafe event has occurred. The expression
of TESLA assertions is similar in concept to this, but with extensions to check
function arguments and return values. MOPS assertions express slightly
different properties to TESLA assertions (and vice versa). MOPS can assert
properties such as \textquote{a call to \mintinline{c}{f} is immediately
followed by a call to \mintinline{c}{g}}, while TESLA can assert that
\textquote{if execution reaches this program point, \mintinline{c}{f} was
previously called}. Broadly, however, the concepts are similar.

The primary advantage of the TESLA model checker compared to MOPS is the
inclusion of assertion site events for considering control flow only on certain
paths, and the ability to check a subset of the program's data flow.

\subsubsection{KLEE}

KLEE \cite{cadar_klee:_2008} is a system for symbolic execution of programs in
order to automatically generate tests or prove assertions. It differs from other
approaches described here as it does not perform model checking---instead, it
generates constraints that must hold for a program point to be reachable, then
solves the constraints to generate test inputs to the program.

The primary motivation for this implementation style is to increase the
source-level coverage of a program's test suite. This means exploring every
possible execution path, an approach that TESLA sought explicitly to avoid (by
using automata bounds and assertion site events). However, some of the ideas
present in KLEE are relevant to the TESLA model checker---like the return value
inference algorithm I describe in \autoref{sec:rvc}, KLEE uses an SMT
translation of LLVM IR to solve constraint systems.

\section{SMT} \label{sec:smt}

In this section I give a brief overview of the theory of SMT methods, as well as
a summary of important work related to program analysis using SMT methods.

\subsection{Background}

As its name suggests, the study of satisfiability modulo theories (SMT) is
founded in boolean satisfiability. The decision problem SAT was the first such
problem to be proven to be NP-complete \cite{cook_complexity_1971}---its
statement can be given concisely as \textquote{Does there exist a consistent
assignment of truth values to the variables in a boolean formula such that the
formula is satisfied?}. SAT exhibits the useful property of self-reducibility,
meaning that any algorithm that solves the decision problem can be used to find
a satisfying assigment.

Many problems can be easily reduced to SAT (formally, any problem in NP can be
reduced to SAT in polynomial time, and informally, its structure makes it a good
choice for encoding some domain-specific problems). However, many other problems
are stated with respect to a background theory such as integer arithmetic or
finite arrays. The key idea of SMT problems is to allow for a \emph{background
theory} to be combined with a satisfiability problem.

\textcite[ch. 12]{biere_handbook_2009} provide a formal definition of SMT
problems, as well as several commonly-used background theories. For the purposes
of this report, only a basic definition is required.

\subsubsection{Terminology}

A \emph{background theory} is a collection of axioms that allow for
interpretation of the symbols in a formula. For example, the background theory
of integer arithmetic provides the standard interpretations of symbols such as $
+, -, \times, 0 $ etc. A different background theory such as that of finite
bit-vectors may interpret these symbols differently (for example, \ $+$ could be
defined to wrap on overflow). Background theories are used because it is often
either tedious or impossible to encode these axioms in propositional logic for a
SAT solver.

\emph{Uninterpreted functions} are the building blocks of SMT instances. No
meaning is associated with these functions when an SMT problem is specified,
only that they have a particular \emph{sort}\footnote{Informally, sorts can be
understood as ``types'' in a particular problem. Examples are the sorts of
integers, booleans, and functions with particular domain and range.}. An SMT
solver may assign interpretations (definitions) to these functions in order to
satisfy the instance constraints.

\subsubsection{Tools and Standards}

A great deal of research and engineering work is invested in the use of SMT
tools. Two of the most commonly used solver implementations are Z3
\cite{de_moura_z3:_2008} and CVC4 \cite{barrett_cvc4_2011}, but there are
numerous others with individual strengths and weaknesses. There exist standards
such as SMT-LIB \cite{BarFT-SMTLIB} that specify textual input and output
formats for SMT solvers. Standardisation in this way allows for competitive
benchmarking of solver performance (the primary venue for this is SMT-COMP
\cite{CDW14}).

\subsection{Related Work}

SMT solvers are a low-level tool---using them to solve a domain-specific problem
involves translating the problem into a formulae in a particular theory. Because
of this flexibility, SMT solvers have been used to solve a large number of
different problems. For example, Microsoft list 58 publications related to
Z3\footnote{\url{https://github.com/Z3Prover/z3/wiki/Publications}} that span
areas as diverse as cloud computing, real-time systems, and functional
programming. Broadly, most applications of SMT methods involve some form of
\emph{program analysis}.

\textcite{dahlweid_vcc:_2009} provide VCC, a tool for proving partial
correctness of C programs using annotations that describe invariants on data
structures. These annotations are converted to an intermediate representation,
then to an SMT problem to be verified. The annotations supported by VCC are
somewhat different to TESLA assertions---they specify invariant rather than
temporal properties. VCC was used successfully to verify the implementation of
the Microsoft Hyper-V kernel.

% Cutting these paragraphs out for not being directly relevant to program
% analysis.
% 
% The interactive theorem prover Isabelle \cite{paulson_isabelle:_2000} supports
% the use of SMT solvers to generate proofs. These proofs are written in a
% higher-level formalism that can then be extracted into verified functional
% programs. This approach is conceptually very different to TESLA and similar
% ``systems'' tools---rather than starting from a program to be verified, the
% proof itself is the primary object of interest.

% Recent work has emerged on the use of SMT solvers as tools for performing
% ``superoptimisation'' (Optgen \cite{buchwald_optgen:_2015} and
% Souper \cite{regehr_souper_2017} are two examples). This approach aims to
% discover potential optimisations that are left unimplemented by a compiler.
% Souper translates LLVM IR into SMT formulae, but is otherwise conceptually
% unrelated to TESLA.

PAGAI \cite{henry_pagai:_2012} uses SMT solvers to implement analyses based on
abstract interpretation. For example, it can be used to discover invariants that
hold at points in the control flow graphs, and to prove properties based on
assertion reachability. The methods used in PAGAI to map LLVM IR onto an SMT
problem are more sophisticated that those I describe in \autoref{sec:rvc}---for
example, PAGAI implements an arithmetic simplification method based on parallel
assignments that allows for stronger invariants to be proved.
