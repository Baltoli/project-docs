In this chapter I provide an overview of related work in areas relevant
to this project, along with a summary of how my work on TESLA
contributes new ideas and developments.

\section{Temporal Logic Assertions} \label{sec:bounded-model-checking}

There exists a good deal of prior research related to program
verification using temporal logic assertions and model checking. In this
section I give a summary of the important work in this area.

\subsection{Bounded Model Checking}

In this section I give a sumamry of previous work in the area of bounded
model checking, with particular emphasis on where it has been applied to
check C programs. I also note key differences between these previous
works and my implementation of a TESLA model checker---namely the style
of assertion supported, and the use of non-symbolic checking in my
implementation.

\subsubsection{BMC}

\textcite{biere_symbolic_1999} introduce the concept of bounded model
checking, building upon the earlier idea of \emph{symbolic} model
checking due to \textcite{mcmillan_symbolic_1992} while removing the
need to construct BDDs\footnote{Binary Decision Diagrams}. Their
implementation of a bounded model checker, BMC, used the state
description language SMV\footnote{In fact, SMV describes both the state
structure and the assertions made of it.} described by
\citeauthor{mcmillan_symbolic_1992} and was able to show significant
performance improvements over previous work.

At the heart of symbolic model checking is the idea of counterexample
generation---if a state is found to satisfy the negation of a formula,
then that state is a counterexample for the original formula. The model
checking process can then be undertaken as a satisfiability problem.
Bounded model checking builds on this idea by searching for
counterexamples with an upper bound on the allowable length. This means
that the counterexamples discovered are of minimal length, and are
discovered faster.

While the implementation strategies and analysis goals of BMC and my
model checker are quite different (BMC is a symbolic model checker for
SMV, my model checker is a non-symbolic model checker for TESLA
assertions embedded in C programs), the strategy of searching for
counterexamples using an iterative-deepening method is derived in part
from the original work on bounded model checking.

\subsubsection{CBMC}

\citeauthor{clarke_behavioral_2003}'s CBMC \cite{clarke_behavioral_2003}
is perhaps the first instance of bounded model checking being applied to
C programs (rather than to a state description language such as SMV).
The focus of the CBMC tool was to allow for C programs to be written as
executable specifications for synchronous Verilog hardware designs
without prohibiting the use of any C programming construct (use of
pointers, memory allocations and recursive calls are all permitted).

Ideas from previous work on BMC together with the new idea of
\emph{unwinding assertions}\footnote{An assertion that captures the idea
that a loop has been unrolled \textquote{enough} times.} allowed C
programs and assertions to be translated into boolean formulae, which
could then be checked using a SAT solver.

Although CBMC is an example of model checking being applied to C
programs, the assertions that can be checked are only local assertions
(albeit with some temporal information available by supplying the value
of Verilog signals to the C program). In this sense, the scope and
target of TESLA assertions is different to those of CBMC---TESLA
assertions are written to verify the behaviour of a program, while CBMC
uses a C program with assertions to verify that a hardware
implementation is behaviourally equivalent to that program.

\subsubsection{LLBMC}

CBMC operates at the level of C source code (the translation to a
boolean formula is specified using syntactic transformations).
\textcite{merz_llbmc:_2012} identified a potential avenue for
improvement upon this model---their LLBMC operates on LLVM
\cite{lattner_llvm:_2002} intermediate representation. The benefits of
this approach include broader language support, assistance from compiler
optimisations for code simplification, and an improved memory model. The
assertions supported by LLBMC are a combination of user-supplied local
assertions, and a set of built-in safety assertions (arithmetic
overflow, memory safety etc.).

Empirically, LLBMC represents an improvement over previous work in the
area it is compared to---it is able to detect more assertion failures in
a larger set of programs than both CBMC and an extended implementation
thereof (ESBMC \cite{cordeiro_smt-based_2009}).

The use of a compiler intermediate representation to simplify model
checking is similar in concept to TESLA (modulo the differences between
symbolic and non-symbolic model checking).

\subsubsection{Context-bounded LTL Checking}

More recent work by \textcite{morse_context-bounded_2011,
morse_model_2015-1} uses bounded model checking to verify more complex
LTL assertions made of a C program. Their approach involves the
translation of an LTL formula into a B\"uchi automaton, which is then
itself converted to C code and woven into the program to be checked.

This approach is reminiscent of TESLA instrumentation, but has some key
differences. The automaton code here has its own thread of execution
rather than being inlined as TESLA instrumentation is, and instead of
producing a modified executable that exhibits runtime failures on
assertion violations, the combined program is checked statically using
ESMBC to check for violations.

Although the development of this work was partly contemporaneous with
the initial work on TESLA, and the implementation strategies were
similar in some respects, the LTL assertions checkable by this system
are closer to those asserted by CBMC than to those asserted by TESLA.

\subsection{Other Approaches}

While there has been a great deal of work derived from BMC on verifying
systems software, there are also other approaches that do not share the
same lineage.

\subsubsection{MOPS}

\section{Path Sensitive Dataflow Analysis}

\section{Protocol Verification}
