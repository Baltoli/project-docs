In this chapter I provide an overview of related work in areas relevant
to this project, along with a summary of how my work on TESLA
contributes new ideas and developments.

\section{Temporal Logic Assertions} \label{sec:bounded-model-checking}

There exists a good deal of prior research related to program
verification using temporal logic assertions and model checking. In this
section I give a summary of the important work in this area.

\subsection{Bounded Model Checking}

\textcite{biere_symbolic_1999} introduce the concept of bounded model
checking, building upon the earlier idea of \emph{symbolic} model
checking due to \textcite{mcmillan_symbolic_1992} while removing the
need to construct BDDs\footnote{Binary Decision Diagrams}. Their
implementation of a bounded model checker, BMC, used the state
description language SMV described by
\citeauthor{mcmillan_symbolic_1992} and was able to show significant
performance improvements over previous work.

At the heart of symbolic model checking is the idea of counterexample
generation---if a state is found to satisfy the negation of a formula,
then that state is a counterexample for the original formula. The model
checking process can then be undertaken as a satisfiability problem.
Bounded model checking builds on this idea by searching for
counterexamples with an upper bound on the length of such examples.

Searching for counterexamples by length has a number of benefits---for
example, they are easier for users of the system to understand and are
faster to generate.

\subsection{CBMC}

\citeauthor{clarke_behavioral_2003}'s CBMC \cite{clarke_behavioral_2003}
is perhaps the first instance of 

\section{Path Sensitive Dataflow Analysis}

\section{Protocol Verification}
