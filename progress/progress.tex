\documentclass[a4paper]{article}

\usepackage[margin=1in]{geometry}
\usepackage{mathpazo}

\begin{document}

\pagenumbering{gobble}

\section*{Static Analysis for TESLA (Progress Report)}

\subsection*{Familiarisation with TESLA}

During the first phase of the project, I worked to develop a TESLA model of a
simple mutex implementation. This allowed me to become familiar with the usage
of TESLA from the perspective of a user, as well as with the implementation,
build process and setup. The assertions I developed during this phase are able
to validate important properties of the usage of a mutex (for example, that it
is not released before being acquired or vice-versa). I also developed a suite
of example programs that demonstrate the cases my assertions are able to cover.

As well as developing these assertions, I made contributions to TESLA in a
larger sense. I was able to discover and fix several bugs in the existing TESLA
code, develop build scripts for integrating TESLA into a project, and collect a
set of notes on using TESLA as a programmer.

\subsection*{Hand-Coded Static Analysis}

The next stage of the project involved the exploration of how the assertions
developed previously could be checked statically (treating them as a special
case, where the properties being asserted are known at development time). The
focus at this step was on optimising TESLA performance by removing
statically-provable assertions. I developed a set of LLVM passes, which aim to
prove a program correct with respect to the properties of my TESLA assertions.

I developed an extension to the TESLA instrumentation stage that runs LLVM
passes, then makes changes to an automata manifest depending on the results (for
example, an assertion can be removed if it is statically provable). The example
programs developed previously were used to validate the correctness of applying
static analysis to the mutex assertions. A reproducible performance improvement
was found on a microbenchmark that caused contention on a TESLA-instrumented
mutex.

\subsection*{Model Checking}

Moving on from checking properties of specific assertions, the next stage was to
generalise to arbitrary assertions that are not known at development time. The
approach I have taken at this stage is to extract an abstract model of program
behaviour from the intermediate representation, then model-check this against a
TESLA assertion.

The algorithm I have implemented to this end falls broadly into the class of
bounded model checking---finite traces of a given length are generated from the
abstract interpretation of the IR, then properties are validated against
individual traces. Because of the way in which TESLA assertions are structured,
the algorithm is somewhat different to pedagogical examples that proceed by
structural induction on a logical formula, and required a carefully considered
design to ensure correctness with respect to the semantics of a TESLA assertion.
As part of the development of this algorithm, I have produced a formalisation of
the properties asserted by statically checkable TESLA assertions.

Currently, the model checker is less powerful than the hand-coded mutex
analyses---properties that rely on data-dependent control flow cannot be checked
statically. However, 80\% of the mutex test programs yield correct results using
the model checker. Adding analysis of data-dependent control flow is likely to
cover the remaining 20\%.

\subsection*{Applications}

In order to find a suitable candidate to evaluate the model checker against, I
developed a simple socket-based client / server protocol for exchanging
messages. Instrumenting the internal state transitions of the server
consistently decreased throughput by 5\%.

Based on this motivating example, I have been working to integrate TESLA into
lwIP (a user-mode network stack implementation). At the time of writing, I have
a FreeBSD port of lwIP that can be compiled using TESLA. The next steps here are
to instrument the library, and demonstrate the application of static analysis to
this instrumentation through the use of a suitable benchmark application.

\end{document}
